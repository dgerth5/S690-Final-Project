% Options for packages loaded elsewhere
\PassOptionsToPackage{unicode}{hyperref}
\PassOptionsToPackage{hyphens}{url}
\PassOptionsToPackage{dvipsnames,svgnames,x11names}{xcolor}
%
\documentclass[
  letterpaper,
  DIV=11,
  numbers=noendperiod]{scrartcl}

\usepackage{amsmath,amssymb}
\usepackage{iftex}
\ifPDFTeX
  \usepackage[T1]{fontenc}
  \usepackage[utf8]{inputenc}
  \usepackage{textcomp} % provide euro and other symbols
\else % if luatex or xetex
  \usepackage{unicode-math}
  \defaultfontfeatures{Scale=MatchLowercase}
  \defaultfontfeatures[\rmfamily]{Ligatures=TeX,Scale=1}
\fi
\usepackage{lmodern}
\ifPDFTeX\else  
    % xetex/luatex font selection
\fi
% Use upquote if available, for straight quotes in verbatim environments
\IfFileExists{upquote.sty}{\usepackage{upquote}}{}
\IfFileExists{microtype.sty}{% use microtype if available
  \usepackage[]{microtype}
  \UseMicrotypeSet[protrusion]{basicmath} % disable protrusion for tt fonts
}{}
\makeatletter
\@ifundefined{KOMAClassName}{% if non-KOMA class
  \IfFileExists{parskip.sty}{%
    \usepackage{parskip}
  }{% else
    \setlength{\parindent}{0pt}
    \setlength{\parskip}{6pt plus 2pt minus 1pt}}
}{% if KOMA class
  \KOMAoptions{parskip=half}}
\makeatother
\usepackage{xcolor}
\setlength{\emergencystretch}{3em} % prevent overfull lines
\setcounter{secnumdepth}{-\maxdimen} % remove section numbering
% Make \paragraph and \subparagraph free-standing
\makeatletter
\ifx\paragraph\undefined\else
  \let\oldparagraph\paragraph
  \renewcommand{\paragraph}{
    \@ifstar
      \xxxParagraphStar
      \xxxParagraphNoStar
  }
  \newcommand{\xxxParagraphStar}[1]{\oldparagraph*{#1}\mbox{}}
  \newcommand{\xxxParagraphNoStar}[1]{\oldparagraph{#1}\mbox{}}
\fi
\ifx\subparagraph\undefined\else
  \let\oldsubparagraph\subparagraph
  \renewcommand{\subparagraph}{
    \@ifstar
      \xxxSubParagraphStar
      \xxxSubParagraphNoStar
  }
  \newcommand{\xxxSubParagraphStar}[1]{\oldsubparagraph*{#1}\mbox{}}
  \newcommand{\xxxSubParagraphNoStar}[1]{\oldsubparagraph{#1}\mbox{}}
\fi
\makeatother

\usepackage{color}
\usepackage{fancyvrb}
\newcommand{\VerbBar}{|}
\newcommand{\VERB}{\Verb[commandchars=\\\{\}]}
\DefineVerbatimEnvironment{Highlighting}{Verbatim}{commandchars=\\\{\}}
% Add ',fontsize=\small' for more characters per line
\usepackage{framed}
\definecolor{shadecolor}{RGB}{241,243,245}
\newenvironment{Shaded}{\begin{snugshade}}{\end{snugshade}}
\newcommand{\AlertTok}[1]{\textcolor[rgb]{0.68,0.00,0.00}{#1}}
\newcommand{\AnnotationTok}[1]{\textcolor[rgb]{0.37,0.37,0.37}{#1}}
\newcommand{\AttributeTok}[1]{\textcolor[rgb]{0.40,0.45,0.13}{#1}}
\newcommand{\BaseNTok}[1]{\textcolor[rgb]{0.68,0.00,0.00}{#1}}
\newcommand{\BuiltInTok}[1]{\textcolor[rgb]{0.00,0.23,0.31}{#1}}
\newcommand{\CharTok}[1]{\textcolor[rgb]{0.13,0.47,0.30}{#1}}
\newcommand{\CommentTok}[1]{\textcolor[rgb]{0.37,0.37,0.37}{#1}}
\newcommand{\CommentVarTok}[1]{\textcolor[rgb]{0.37,0.37,0.37}{\textit{#1}}}
\newcommand{\ConstantTok}[1]{\textcolor[rgb]{0.56,0.35,0.01}{#1}}
\newcommand{\ControlFlowTok}[1]{\textcolor[rgb]{0.00,0.23,0.31}{\textbf{#1}}}
\newcommand{\DataTypeTok}[1]{\textcolor[rgb]{0.68,0.00,0.00}{#1}}
\newcommand{\DecValTok}[1]{\textcolor[rgb]{0.68,0.00,0.00}{#1}}
\newcommand{\DocumentationTok}[1]{\textcolor[rgb]{0.37,0.37,0.37}{\textit{#1}}}
\newcommand{\ErrorTok}[1]{\textcolor[rgb]{0.68,0.00,0.00}{#1}}
\newcommand{\ExtensionTok}[1]{\textcolor[rgb]{0.00,0.23,0.31}{#1}}
\newcommand{\FloatTok}[1]{\textcolor[rgb]{0.68,0.00,0.00}{#1}}
\newcommand{\FunctionTok}[1]{\textcolor[rgb]{0.28,0.35,0.67}{#1}}
\newcommand{\ImportTok}[1]{\textcolor[rgb]{0.00,0.46,0.62}{#1}}
\newcommand{\InformationTok}[1]{\textcolor[rgb]{0.37,0.37,0.37}{#1}}
\newcommand{\KeywordTok}[1]{\textcolor[rgb]{0.00,0.23,0.31}{\textbf{#1}}}
\newcommand{\NormalTok}[1]{\textcolor[rgb]{0.00,0.23,0.31}{#1}}
\newcommand{\OperatorTok}[1]{\textcolor[rgb]{0.37,0.37,0.37}{#1}}
\newcommand{\OtherTok}[1]{\textcolor[rgb]{0.00,0.23,0.31}{#1}}
\newcommand{\PreprocessorTok}[1]{\textcolor[rgb]{0.68,0.00,0.00}{#1}}
\newcommand{\RegionMarkerTok}[1]{\textcolor[rgb]{0.00,0.23,0.31}{#1}}
\newcommand{\SpecialCharTok}[1]{\textcolor[rgb]{0.37,0.37,0.37}{#1}}
\newcommand{\SpecialStringTok}[1]{\textcolor[rgb]{0.13,0.47,0.30}{#1}}
\newcommand{\StringTok}[1]{\textcolor[rgb]{0.13,0.47,0.30}{#1}}
\newcommand{\VariableTok}[1]{\textcolor[rgb]{0.07,0.07,0.07}{#1}}
\newcommand{\VerbatimStringTok}[1]{\textcolor[rgb]{0.13,0.47,0.30}{#1}}
\newcommand{\WarningTok}[1]{\textcolor[rgb]{0.37,0.37,0.37}{\textit{#1}}}

\providecommand{\tightlist}{%
  \setlength{\itemsep}{0pt}\setlength{\parskip}{0pt}}\usepackage{longtable,booktabs,array}
\usepackage{calc} % for calculating minipage widths
% Correct order of tables after \paragraph or \subparagraph
\usepackage{etoolbox}
\makeatletter
\patchcmd\longtable{\par}{\if@noskipsec\mbox{}\fi\par}{}{}
\makeatother
% Allow footnotes in longtable head/foot
\IfFileExists{footnotehyper.sty}{\usepackage{footnotehyper}}{\usepackage{footnote}}
\makesavenoteenv{longtable}
\usepackage{graphicx}
\makeatletter
\def\maxwidth{\ifdim\Gin@nat@width>\linewidth\linewidth\else\Gin@nat@width\fi}
\def\maxheight{\ifdim\Gin@nat@height>\textheight\textheight\else\Gin@nat@height\fi}
\makeatother
% Scale images if necessary, so that they will not overflow the page
% margins by default, and it is still possible to overwrite the defaults
% using explicit options in \includegraphics[width, height, ...]{}
\setkeys{Gin}{width=\maxwidth,height=\maxheight,keepaspectratio}
% Set default figure placement to htbp
\makeatletter
\def\fps@figure{htbp}
\makeatother

\KOMAoption{captions}{tableheading}
\makeatletter
\@ifpackageloaded{caption}{}{\usepackage{caption}}
\AtBeginDocument{%
\ifdefined\contentsname
  \renewcommand*\contentsname{Table of contents}
\else
  \newcommand\contentsname{Table of contents}
\fi
\ifdefined\listfigurename
  \renewcommand*\listfigurename{List of Figures}
\else
  \newcommand\listfigurename{List of Figures}
\fi
\ifdefined\listtablename
  \renewcommand*\listtablename{List of Tables}
\else
  \newcommand\listtablename{List of Tables}
\fi
\ifdefined\figurename
  \renewcommand*\figurename{Figure}
\else
  \newcommand\figurename{Figure}
\fi
\ifdefined\tablename
  \renewcommand*\tablename{Table}
\else
  \newcommand\tablename{Table}
\fi
}
\@ifpackageloaded{float}{}{\usepackage{float}}
\floatstyle{ruled}
\@ifundefined{c@chapter}{\newfloat{codelisting}{h}{lop}}{\newfloat{codelisting}{h}{lop}[chapter]}
\floatname{codelisting}{Listing}
\newcommand*\listoflistings{\listof{codelisting}{List of Listings}}
\makeatother
\makeatletter
\makeatother
\makeatletter
\@ifpackageloaded{caption}{}{\usepackage{caption}}
\@ifpackageloaded{subcaption}{}{\usepackage{subcaption}}
\makeatother

\ifLuaTeX
  \usepackage{selnolig}  % disable illegal ligatures
\fi
\usepackage{bookmark}

\IfFileExists{xurl.sty}{\usepackage{xurl}}{} % add URL line breaks if available
\urlstyle{same} % disable monospaced font for URLs
\hypersetup{
  pdftitle={Effects of Surgery on Chimp Memory},
  pdfauthor={David Gerth},
  colorlinks=true,
  linkcolor={blue},
  filecolor={Maroon},
  citecolor={Blue},
  urlcolor={Blue},
  pdfcreator={LaTeX via pandoc}}


\title{Effects of Surgery on Chimp Memory}
\author{David Gerth}
\date{}

\begin{document}
\maketitle


\subsection{Executive Summary}\label{executive-summary}

\begin{itemize}
\tightlist
\item
  Goal: Estimate the effect of a particular surgery on chimp memory.\\
\item
  Methods: A binomial logistic regression model was fit with terms for
  time and if the chimp had the surgery or not, along with an
  interaction between the two.\\
\item
  Results: There was a significant difference in scores between chimps
  that had surgery and those that did in the 2-4 weeks after the surgery
  date, but the difference in scores between groups diminishes over the
  long run.
\end{itemize}

\subsection{Introduction}\label{introduction}

The goal of the study is to determine if a particular surgery affects
the memory of chimps who are taking a test. For each chimp, they are
shown two objects and need to choose the object with the raisin
underneath. Each chimp goes through this 20 times, and the percentage of
correct choices is recorded. Each chimp is tested at the time interval
two weeks after surgery date, four weeks, eight weeks, 12 weeks and 16
weeks.

For this analysis, we want to have a methodology that estimates if the
surgery has an effect on chimp memory, and what is the magnitude
(measured by difference in scores between the control and treated group)
of the effect of surgery if there is one. Since this study is being
conducted over time, we also want to know how time affects the
significance and magnitude of surgery.

\subsection{Description of Dataset}\label{description-of-dataset}

The dataset consists of 18 chimps, each with their own name and ID.
Seven of them were in the control group while the remaining 11 were in
the ``treated'' group. As discussed in the introduction, each chimp is
evaluated at the time period of 2, 4, 8, 12, and 16 weeks post surgery
date, with the percentage of correct choices recorded. In the plot
below, you can see how each chimp scored over time.

\includegraphics{chimp-scores-over-time.png}

The red line here indicates the split between short-term (2 and 4 weeks)
and long term (8, 12, and 16 weeks). This will be discussed more in the
Methodology section.

There were no missing values in the dataset, so this was a
straightforward dataset to analyze from a missing data perspective.

\subsection{Methodology}\label{methodology}

Many different model types were discussed, but the model ultimately
selected was a binomial logistic regression, where the outcome variable
was the number of correct choices out of 20.

In this model, there are two variables: Is\_Short\_Term, which is 1 if
Week is either 2 or 4, 0 otherwise, and Treatment, which is a factor
with values ``treated'' and control.'' An interaction term is used to
estimate if the effect of treatment changes based on what time period
the chimp is in.

Leaving treatment as a factor was straightforward, but the time
component was tested in a few different ways. Treating week as a factor,
treating week as numeric, and using a spline for time were all
considered, but comparing models by BIC, the binomial logistic
regression with a binary short term / long term variable and a variable
for the chimp's treatment had the lowest BIC, and was easiest to
interpret while directly answering the client's question.

In mathematical terms, the model can be written as:

\[
\begin{aligned}
\text{TotalCorrect}_i &\sim \text{Binomial}(n = 20,\, p_i) \\
\text{logit}(p_i) &= \beta_0
+ \beta_1 \, \text{Is\_ST}_i
+ \beta_2 \, \text{Treatment}_i
+ \beta_3 \, (\text{Is\_ST}_i \times \text{Treatment}_i)
\end{aligned}
\]

In R:

\begin{Shaded}
\begin{Highlighting}[]
\NormalTok{model }\OtherTok{\textless{}{-}} \FunctionTok{glm}\NormalTok{(}\FunctionTok{cbind}\NormalTok{(total\_correct, trials }\SpecialCharTok{{-}}\NormalTok{ total\_correct) }\SpecialCharTok{\textasciitilde{}}\NormalTok{ is\_ST }\SpecialCharTok{*}\NormalTok{ Treatment,}
               \AttributeTok{data =}\NormalTok{ chimp\_data\_reshape,}
               \AttributeTok{family =} \FunctionTok{binomial}\NormalTok{())}
\end{Highlighting}
\end{Shaded}

An extension of the model that was tested was adding a random effect for
chimp name. It seemed natural to me that treatment effect would vary by
chimp, or that there was innate differences in memory for each chimp.
This was tested in with a few different formulations of random effects
structures by the code below:

\begin{Shaded}
\begin{Highlighting}[]
\NormalTok{mixed\_model1 }\OtherTok{\textless{}{-}} \FunctionTok{glmer}\NormalTok{(}\FunctionTok{cbind}\NormalTok{(total\_correct, trials }\SpecialCharTok{{-}}\NormalTok{ total\_correct) }\SpecialCharTok{\textasciitilde{}}\NormalTok{ is\_ST }\SpecialCharTok{*}\NormalTok{ Treatment }\SpecialCharTok{+}\NormalTok{ (}\DecValTok{1}\SpecialCharTok{|}\NormalTok{Monkey),}
                      \AttributeTok{data =}\NormalTok{ chimp\_data\_reshape,}
                      \AttributeTok{family =} \FunctionTok{binomial}\NormalTok{())}
\CommentTok{\# extract ranef\textquotesingle{}s: all 0 bc singular}
\FunctionTok{ranef}\NormalTok{(mixed\_model1)}
\CommentTok{\# confidence intervals for parameters}
\FunctionTok{confint}\NormalTok{(mixed\_model1, }\AttributeTok{method =} \StringTok{"boot"}\NormalTok{)}
\end{Highlighting}
\end{Shaded}

The mixed effects model gives a singular fit, indicating that the random
effect variance estimated is 0, and that the random effect for chimp is
not needed. This is further confirmed by estimating parametric bootstrap
confidence intervals for each parameter, and the confidence interval for
the random effect variance includes 0. The parametric bootstrap is used
here because the distribution of the random effect variance is skewed
and fixed effect estimates depend on random effect estimates, which
violates how a typical confidence interval is derived for typical
linear/logistic model, hence why a bootstrap approach is used.
(Extending the Linear Model, Faraway, pages 180-185)

Overall, the final model is simple, but allows us to answer if the
surgery has an effect on chimp memory, and to what degree.

\subsection{Results}\label{results}

Below is the table of coefficients for the binomial logistic regression.

\textbf{Coefficients:}

\begin{longtable}[]{@{}lll@{}}
\toprule\noalign{}
Coefficient & Value & p value \\
\midrule\noalign{}
\endhead
\bottomrule\noalign{}
\endlastfoot
Intercept & 0.73632 & 1.65e-12 \\
Is\_Short\_Term & 0.67245 & 0.000239 \\
Treatment & 0.03281 & 0.806127 \\
Is\_Short\_Term:Treatment & -0.90156 & 5.80e-05 \\
\end{longtable}

From the table, we can see that the interaction between Is\_Short\_Term
and Treatment has a significant p-value at the \(\alpha = 0.05\) level.
Treatment being insignificant on its own means that during the long-term
time frame (Is\_Short\_Term = 0), there is not a detectable difference
between Treatment and Scores. Treatment is important due to the
significant interaction between Time Frame and Treatment, but the effect
of Treatment is different depending on time.

While the coefficient outputs are important to see and provide important
information, the interaction between time frame and treatment type is
easier to understand when looking at the predicted outcomes for each
interaction type, which are summarized in the tables below.

\textbf{Predicted Probabilities:}

\begin{longtable}[]{@{}llll@{}}
\toprule\noalign{}
Is\_ST & Control & Treated & Difference \\
\midrule\noalign{}
\endhead
\bottomrule\noalign{}
\endlastfoot
Yes & 80\% & 63\% & 17\% \\
No & 68\% & 68\% & 0\% \\
\end{longtable}

\textbf{Predicted Log Odds:}

\begin{longtable}[]{@{}lll@{}}
\toprule\noalign{}
Is\_ST & Control & Treated \\
\midrule\noalign{}
\endhead
\bottomrule\noalign{}
\endlastfoot
Yes & 1.409 & 0.540 \\
No & 0.736 & 0.769 \\
\end{longtable}

From the table, you can see that for chimps in the control group, during
the short time frame they are expected to choose the right object around
80\% percent of the time. This figure decreases to 68\% during the long
time frame, indicating that as time increases, their accuracy becomes
worse. In comparison, chimps who were treated also average around 68\%
accuracy in classifying objects over the long term, but during the short
term, they are expected to choose the correct object around 63\% of the
time, which is 17\% lower than for the control group. This is a large
difference and shows that the difference between chimps who had surgery
and those who didn't is substantial over the course of the first 2-4
weeks post surgery.

\subsection{Conclusion}\label{conclusion}

In summary, we have built a model to predict how many times a chimp will
chose the correct object with terms for if the chimp had surgery or not,
and how long after the surgery the event took place. The model shows a
statistically significant interaction between time frame after the
surgery, and whether or not the chimp had the surgery or not. Over the
long term (6+ weeks), the effect of surgery is minimal, but for the
first 2-4 weeks, there is a large difference between scores of chimps
who had surgery versus those who do not. With this model formulation, we
were able to estimate if there was a statistically significant effect
(there is), and also what the magnitude of the effect was (minimal
during short time frame, \textasciitilde17\% difference between treated
and control over long time frame). We also tested to see if there was
any chimp-specific variation by using the mixed effects model, which
there was not.

GitHub repo with full code:
https://github.com/dgerth5/S690-Final-Project/tree/main




\end{document}
